
\chapter{Technical overview}

\section{System architecture}

Skedge is a \emph{Ruby on Rails} web application running on a webserver called \emph{unicorn}, and routes ports 80 to it using the \emph{nginx} webserver as a reverse-proxy. Skedge uses \emph{PostgreSQL} as database, and \emph{React.js} on the front-end, a framework developed by Facebook for making responsive user interfaces. This stack is hosted on a paid (by me) virtual machine instance on DigitalOcean. 

\section{Data storage}

\subsection{Courses}

At the end of each day, Skedge runs an update script that updates its course data with any new course data that exists or has changed on CDCS, endearingly called ``scraping.'' There is minimal text processing to get the data to its final form as it is on Skedge, such as title formatting into proper capitalization. Highly sensitive data such as times, location, or virtually any other field remains completely unmodified. A substianal amount of logic goes into connecting courses to their subsections (workshops, labs, etc.), as these are not explicitly connected in CDCS and sometimes requires some basic natural language processing (finding CRN numbers, title similarity, etc.) Any ambiguities in this process are reported to me and they can easily be resolved (most arise from data entry errors in CDCS).

\subsection{Users \& schedules}

Users are purposefully light-weight on Skedge, because it aims for as little friction and overhead as possible before users can begin to search for courses and build a schedule. Thus there are no logins (besides now with Skedge Social, which is optional), and yet Skedge keeps users' schedules persistent. This is done using a single cookie on the client side that records the user ID that client is associated with, as well as a long hexadecimal string that serves as the ``user secret,'' or ``password'', which must match the one on the server for any changes to be made. This prevents users from manually changing cookie data to modify others' schedules.

The disadvantage of the cookie-based system is that each new browser or device is disjoint by default. Previously, I made it possible to retrieve and set the user secret from the user interface, allowing users to manually synchronize all of their devices' browser cookies to the same user ID and thus manage the same schedules from different devices, but this was replaced by the same feature offered through Skedge Social, as part of promoting the platform. A different way to solve the problem of better persistence while not sacrificing the trade-off of lightweight accounts, and one that doesn't require external services like Facebook will be looked into. A likely candidate at the moment is using University of Rochester NetID for schedule synchronization and improved persistence.

\subsection{Skedge Social}

Out of respect for privacy and a preference to \emph{not} store sensitive data, Skedge does not keep or store any Facebook data, not even users' names or lists of friends, the two pieces of data Skedge requests access to during the integration process. Skedge only stores an anonymous ID (uniquely generated for Skedge by Facebook) per account. Only a user’s name and list of friends also using Skedge are accessible via this ID, and Skedge doesn’t even store this in its database---both are used exclusively by the client, however---separate requests are made for if any given users (friends of the user) are taking certain classes, etc., and for when displaying the schedule feed.

Course \emph{likes} are also entirely locally stored and have nothing to do with Facbeook---Skedge merely uses the same concept and a similar icon for its own feature. Comments on courses are not currently supported, although Facebook-integrated comments would be possible.

\section{Usage data collection}

As the saying goes, ``if it's not monitored, it's not in production'' (how else would we know that it was?) \cite{monitored}.

Skedge collects usage data that is anonymous with regards to who a specific user is (because that information is not even known), but that is not anonymous with respect to a trackable user ID. Keeping track of certain properties across users is very important for the data analysis in Sections 4.2 and 4.3. This tracking is done by the open-source library \emph{Ahoy}, which also has a JavaScript library to log click events unobtrusively and efficiently (e.g. send them after the page has been loaded) using cookies. Note that this has nothing to do with advertisement tracking that is done by marketing agencies across the web---this tracking is entirely local to Skedge and Skedge does not advertise in any way.

\emph{Google Analytics} is also enabled on Skedge, which also involves cookie-based tracking, and is used for higher-level analytics such as session duration, number of sessions, etc. (a session is defined by a period of activity expiring after 30 minutes of inactivity.)