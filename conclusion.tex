
\chapter{Conclusion}

\vspace{-20pt}
\section{Proposal to the University}

On behalf of myself and the roughly two thousand students who regularly and voluntarily choose to use Skedge, I urge the University to consider adopting it as an official replacement to CDCS.

Given the excellence of the University's inspiring faculty and academic programs, I know that I am not alone in my disappointment with what I consider a low regard for usability in many student tools, as well as the lack of communication channels with students to improve them. Throughout the course of this paper, I have shown that the space of course scheduling holds vast potential for innovation, and that CDCS barely scratches that surface. An officially adopted Skedge---or an otherwise substantial overhaul of CDCS---would demonstrate that the University pays attention to its students and to how they use the tools it provides them.

While functional, CDCS nevertheless degrades student experience on campus and discolors perception of the administration's care for students, especially when modern, relevant, and engaging tools are within arm's reach, and most especially at a time when University of Rochester's very high tuition rates continue to rise year-to-year.

As a graduating senior, I cannot commit to continuing the upkeep of server costs and system maintenance for Skedge in the long term, and certainly not indefinitely. For the sake of the University's students, parents, faculty, and staff, I hope that my efforts with Skedge were not in vain and can lead to a substantial refresh of University tools that, most crucially, \textbf{involves a dialogue between the administration, students, and staff}. Thank you for reading.

\\
\vspace{10pt}

\noindent (The views expressed in this section are entirely my own and were not reviewed by my thesis advisor or any others.)

\section{Resources}

\subsection*{Source code}

The source code for Skedge and the scripts used to generate the analytics in Chapter 4 are available online under an open source license:\\
\url{https://github.com/RocHack/skedge}.

\subsection*{Live site}

\noindent The site can be found at \url{http://skedgeur.com} or \url{http://skedge.org} (very recently acquired, simply redirects to the former).


\section{Acknowledgments}

I would like to give my foremost thanks to Dr. Philip Guo, who very thoughtfully reached out to \emph{me} about using Skedge for research, and whose deep care and support for the project has since reignited and bloomed its development. I also want to give enormous thanks to Ani and Mark Gabrellian for their generous Mesrob Mashtots Innovation Grant donation which funded Skedge's development in the summer of 2015. Finally, a huge thanks to all those students and my close friends who reported bugs, gave feedback, and in general supported, promoted, got excited about, and complained about Skedge!