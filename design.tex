
\chapter{Design as a reaction to CDCS}

From its very inception, Skedge's functionality and visual design were driven by the shortcomings of CDCS. It was built \emph{bottom-up}, not \emph{top-down}, as every aspect of the application was either made as a reaction to a particular grievance in CDCS or as the natural evolution of an existing feature. Skedge is thus rooted in \emph{usability} derived from real need, not mere conjecture along the question ``what could students want?''. Its success with students, shown in Chapter 4, demonstrates that this usability extends beyond my own standard and can fulfill the various discovered use-cases of students in general.

In this chapter, I will

%%%

\section{Modernity}

CDCS is an old system.

\subsection{GET requests vs. AJAX}

- Can use back button
- Can send a link to a course or search

\subsection{Built-in scheduler vs. browser extension}

- Better UX
- Data is centralized

\subsection{Mobile}

- Important nowadays
- Mobile traffic stat or smth

\subsection{Public API}

- Important nowadays, extends student possibility
- JSON
- Brief demo of API

%%%

\section{Usability}

\subsection{Data quality}

- Courses don't shout
- Typos in comments
- 12-hour time

\subsection{Section display}

- Grouped course sections
- Embedded labs (A/B too), workshops, \& recitations

\subsection{Course reference}

- Clickable/hoverable course links, professor searches

\subsection{Multiple schedule support}

- Old CDCS+betterCDCS system can't keep track of this, have conflicts when adding stuff

\subsection{Exporting to GCal, .ics, image}

- Mobile sync support
- Security: BetterCDCS export gcal is currently broken and sends netID in PLAINTEXT over http(!!!)

\subsection{Search}

Most important usability concern is finding courses.

%%%

\section{Search}

Use cases, natural language.

\subsection{Course selection criteria}

Narrowed it down to three criteria. Keep in mind that \emph{none} of the things listed below are supported by CDCS, and they are all supported by Skedge.

  \subsubsection{Requirements}

  - Finding crosslists
  - Clusters

  \subsubsection{Browsing}

  - ``New'' courses
  - ``Autofit'' search
  - Random
  - Sorts

  \subsubsection{Friends}

  - ``What are my friends taking?'' (``what are you taking this semester'' = probably most common smalltalk phrase uttered on campus)
  - ``What do my friends recommend?'' - ``have you taken this class, and if so, what did you think of it?''

\subsection{Natural language search}

%Figure

See figure.

  \subsubsection{Advantages}

  - 15 fields reduced to 1

  vs form entry:
  - Faster
  - More intuitive
  - More easily extendable

  \subsubsection{Disadvantages}

  Having to know the DSL, grammar ambiguities (can be solved with a `did you mean')

\subsection{Multipurpose}

Used by other links (instructors, course references) around the site

\subsection{Added features}

- CRN (!)
- Crosslist
- Class size

%%%

\section{Social}

\subsection{The issue}

  \subsubsection{Static image vs. live site}

  - Edits don’t update
  - Referencing courses

  \subsubsection{Finding common courses}

  - requires your friends to share their schedules on FB publicly and you to see their post


%%%%%%%%%%%%%%


  - is schedule-first, not search-first
  - typically only occurs for the current semester

%figure

\subsection{Skedge Social}

% Walkthru

  \subsubsection{Friends' course enrollments}

  Mini-feed

  \subsubsection{Friends' course likes}

  \subsubsection{Likes \& enrollments embedded in results}

  \subsubsection{Personal schedule synchronization}

  \subsubsection{Privacy}

  \subsubsection{Notifications}

  %figs of the 2 types