
\chapter{Design in reaction to CDCS}

From its very inception, Skedge's functionality and visual design were driven by the shortcomings of CDCS. Skedge is built \emph{bottom-up}, not \emph{top-down}---every aspect of the application was either made as a reaction to a particular grievance in CDCS or as the natural evolution of an existing feature. Skedge is thus rooted in \emph{usability} derived from real need, not mere conjecture along the question ``what could students want?''. Its success with students, shown in Chapter 4, demonstrates that this usability extends beyond my own standard and can fulfill the various discovered use-cases of students in general.

In this chapter, I invite the reader along on a tour of these grievances and their remedies.

%%%

\input{design-modernity}
\clearpage

%%%

\input{design-usability}
\clearpage

%%%

\input{design-search}
\clearpage

%%%

\input{design-social}