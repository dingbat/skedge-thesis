\documentclass[titlepage]{report}

\usepackage{setspace}

\usepackage[utf8]{inputenc}
\usepackage{amsmath}   
\usepackage{mathtools}
\usepackage{graphicx}
\usepackage{subcaption}
\usepackage{float}
\usepackage{hyperref}
\usepackage{listings}
\usepackage[labelfont=bf]{caption}
\usepackage[margin=3.5cm]{geometry}

\lstset{
	basicstyle=\footnotesize\ttfamily,
}

\title{
\vspace{60pt}\\
\huge \bfseries SKEDGE
\\
\vspace{10pt}
\Large
Smarter course scheduling for our\\
University of Rochester
}

\author{
	Dan Hassin\\
    \vspace{5pt}\\
    Supervised by\\
    Professor Philip Guo\\
    \vspace{2pt}\\
    Department of Computer Science\\
    University of Rochester\\
    Rochester, New York\\
}

\date{April 12, 2016\\
    \vspace{150pt}\\
    submitted in partial fulfillment of\\
    the requirements for the degree\\
    \emph{honors bachelor of science}\\
}

\begin{document}

\maketitle

%%%%%%%%%%%

\onehalfspacing

\pagenumbering{gobble}
\setcounter{tocdepth}{1}
\tableofcontents

\listoftables

\listoffigures

\clearpage

%%%%%%%%%%%%%%%%%%

\doublespacing

\addcontentsline{toc}{chapter}{Abstract}

\begin{abstract}

\thispagestyle{plain}
\pagenumbering{roman}

In this paper I present Skedge, a web application for students to comfortably and effectively engage with the University's course catalog. Skedge matches and surpasses the capabilities of the existing University tool for this purpose, ``Course Description / Course Schedule'' (CDCS) and presents its information in a more visually pleasing way. As a result, Skedge boasts strong user-retention rates, long session durations, and high student adoption despite having virtually no advertisement. Through collected usage data, I demonstrate that a) Skedge's differences from and additions to CDCS are usable and have real need, b) the two major use-cases associated with course browsing---direct search and exploratory search---are effectively accommodated by Skedge, and c) Skedge's search mechanism is user-friendly and self-teaches to users over time.

\end{abstract}


%%%%%%%%%%%%%%%

\pagenumbering{arabic}



\chapter{Introduction}


\onehalfspacing
{\small
I am proud to finally and officially introduce \emph{Skedge}, a website I have been developing since December 2013. Skedge began as a weekend project after far too much time was spent trying to find interesting elective classes, and it wasn't until I briefly presented it at a RocHack ``Hacker Night'' when I realized the potential it had to help others as well. 
I never expected the subject of my senior thesis to be something so close to my heart, so I am grateful for such a fun and motivating opportunity to continue work on Skedge, and, selfishly, for an outlet of recognition.
}

\doublespacing

\section{CDCS}

\begin{figure}
    \centering
        \begin{subfigure}[h]{14cm}
            \centering
            \fbox{
                \includegraphics[width=1.00\textwidth]{images/cdcs/index}
            }
            \caption{CDCS, with the search query {\tt csc}}
            \label{fig:cdcs-index}
        \end{subfigure}\\
        \vspace{10pt}\\
        \begin{subfigure}[h]{14cm}
            \centering
            \fbox{
                \includegraphics[width=1.00\textwidth]{images/cdcs/better}
            }
            \caption{Better CDCS, a separate browser extension that embeds buttons into the CDCS course results interface, allowing users to add courses to a locally-stored schedule}
            \label{fig:cdcs-better}
        \end{subfigure}
    \caption{CDCS and Better CDCS in their current states}
\end{figure}


``Course Description / Course Schedule,'' (or \emph{CDCS} for short) is the University's official tool for browsing the course catalog \cite{cdcs}. The CDCS interface consists of fields on the left side for inputting a search, and search results on the right side (see Figure \ref{fig:cdcs-index}). Despite ``Course Schedule'' being in its name, CDCS does not offer scheduling functionality. Course times can only be viewed, and it is up to the user to keep track of courses considered for the next semester.

\subsection{``Better CDCS''}

\emph{Better CDCS} \cite{better-cdcs} is a browser extension, available for Firefox, Safari, and Chrome, that offers a solution to the lack of scheduling mentioned above. It works by injecting ``Add Section'' and ``Bookmark Section'' buttons into the existing CDCS interface, and providing a tab on top of the screen that will toggle between the user's schedule and the search results (see Figure \ref{fig:cdcs-better}).

%%%

\section{Skedge}

\begin{figure}[H]
    \centering
    \fbox{
        \includegraphics[width=1.00\textwidth]{images/skedge/index}
    }
    \caption[Skedge with the search query {\tt csc}]{Skedge with the search query {\tt csc} and the user's current schedule on the right}
    \label{fig:sk-index}
\end{figure}

\emph{Skedge} \cite{skedge} is, in short, CDCS combined with Better CDCS on a single platform (see Figure \ref{fig:sk-index}). Like CDCS, it is lightweight, service-oriented, and doesn't require login, but unlike Better CDCS, it also doesn't require a separate browser extension.

It matches feature-for-feature with the two tools (search options, information displayed, scheduling and bookmarks, etc.), and aims to improve upon the work of both on several fronts, which will be investigated in this paper in great detail. It is my contention that students, parents, faculty, staff, and administration can all benefit from such improvements.

\vspace{30pt}

\noindent
This paper is organized into two parts, with a brief intermission in between. First, I will explain many of Skedge's design decisions in response to what I call ``grievances'' with CDCS. We will break for a technical overview, and the second part will look at live usage data, extrapolating how Skedge is used by students and measuring its efficacy as a platform using some novel metrics.
\clearpage


\chapter{Design as a reaction to CDCS}

Improvements were made by USING CDCS, (bottom-up, not top-down)!

%%%

\section{Modernity}

CDCS is an old system.

\subsection{GET requests vs. AJAX}

- Can use back button
- Can send a link to a course or search

\subsection{Built-in scheduler vs. browser extension}

- Better UX
- Data is centralized

\subsection{Mobile}

- Important nowadays
- Mobile traffic stat or smth

\subsection{Public API}

- Important nowadays, extends student possibility
- JSON
- Brief demo of API

%%%

\section{Usability}

\subsection{Data quality}

- Courses don't shout
- Typos in comments
- 12-hour time

\subsection{Section display}

- Grouped course sections
- Embedded labs (A/B too), workshops, \& recitations

\subsection{Course reference}

- Clickable/hoverable course links, professor searches

\subsection{Multiple schedule support}

- Old CDCS+betterCDCS system can't keep track of this, have conflicts when adding stuff

\subsection{Exporting to GCal, .ics, image}

- Mobile sync support
- Security: BetterCDCS export gcal is currently broken and sends netID in PLAINTEXT over http(!!!)

\subsection{Search}

Most important usability concern is finding courses.

%%%

\section{Search}

Use cases, natural language.

\subsection{Course selection criteria}

Narrowed it down to three criteria. Keep in mind that \emph{none} of the things listed below are supported by CDCS, and they are all supported by Skedge.

  \subsubsection{Requirements}

  - Finding crosslists
  - Clusters

  \subsubsection{Browsing}

  - ``New'' courses
  - ``Autofit'' search
  - Random
  - Sorts

  \subsubsection{Friends}

  - ``What are my friends taking?'' (``what are you taking this semester'' = probably most common smalltalk phrase uttered on campus)
  - ``What do my friends recommend?'' - ``have you taken this class, and if so, what did you think of it?''

\subsection{Natural language search}

%Figure

See figure.

  \subsubsection{Advantages}

  - 15 fields reduced to 1

  vs form entry:
  - Faster
  - More intuitive
  - More easily extendable

  \subsubsection{Disadvantages}

  Having to know the DSL, grammar ambiguities (can be solved with a `did you mean')

\subsection{Multipurpose}

Used by other links (instructors, course references) around the site

\subsection{Added features}

- CRN (!)
- Crosslist
- Class size

%%%

\section{Social}

\subsection{The issue}

  \subsubsection{Static image vs. live site}

  - Edits don’t update
  - Referencing courses

  \subsubsection{Finding common courses}

  - requires your friends to share their schedules on FB publicly and you to see their post
  - is schedule-first, not search-first
  - typically only occurs for the current semester

%figure

\subsection{Skedge Social}

% Walkthru

  \subsubsection{Friends' course enrollments}

  Mini-feed

  \subsubsection{Friends' course likes}

  \subsubsection{Likes \& enrollments embedded in results}

  \subsubsection{Personal schedule synchronization}

  \subsubsection{Privacy}

  \subsubsection{Notifications}

  %figs of the 2 types
\clearpage


\chapter{Technical overview}

\section{Back-end}

\section{Front-end}

\section{Analytics}
\clearpage


\chapter{Data Analytics}

Hypotheses:

1. Skedge's differences from and additions to CDCS are usable and have real need

2. Skedge’s navigations-per-add and other metrics demonstrate effectiveness of the use cases
a) direct searching, and
b) course browsing

3. Skedge’s DSL is user-friendly; users learn more advanced search types over time by using it

\section{Usage}

\subsection{General}

Since November 3rd 2015 (137 days)
3,768 unique users
4,500 schedules
Average 90 sessions/day
Average 4.92 pages/session
Average 5:31 minutes/session
28\% of sessions are from new users

MOBILE RESULT

\subsection{Search}

% figs

  \subsubsection{Empty searches}

  Can learn from these
  Some funny ones

\subsection{Course blocks}

40\% of sessions have at least one block-click
Average of 4.94 block-clicks per session

\subsection{Social}

90 users have linked Skedge to Facebook
Since March 1st,
4,000+ visits (200 visits/day)
~60\% of visits to /social were returning visitors
90 overlays onto friends’ schedules
10 clicks to Facebook profiles :(
- get stats from the fb dashboard

\subsection{Conclusion}

Success! Considering skedge is OPTIONAL.
+ course blocks (obv usecase, can't click)
+ exports (not supported by thing)
+ mobile

\section{Navigations-per-add}

\subsection{Definitions}

A navigation is defined as
a search, or
a click on an instructor’s name, or
a click on a crosslisted or prerequisite course link

The navigations-per-{add, bookmark} measure is
the number of navigations a user took (within one session) until a course was {added, bookmarked}

\subsection{Trends}

%figs

\subsection{Breaking them apart}

  behavioral patterns
  Direct search for specific course
  Discovery, browsing, exploring

  \subsubsection{Direct searches}

  \subsubsection{Browse}

  % STAT: Find major of users, find how many non-major courses they found

\subsection{Conclusion}

%figs

Effective++

\section{Users' search types over time}

\subsection{Definitions}

Points for search by (omits number and dept.):

description
credits
crosslisted
CRN
instructor
title
year
term
‘random’
upper-level writing

“CSC” → 0
“MTH 165” → 0
“taught by hema” → 1   ✓    (2 searches) 
“random mur 1-2 credits” → 2   ✓    (1 search) 


\subsection{Trends}

% fig

First increase (60.5% of users)
Median: 2 searches
Average: 4.23 searches
(Starting at 1 counts as an increase value of 0)

Second increase (7.9% of users)
Median: 8 searches
Average: 17.52 searches

\subsection{Conclusion}

DSL++
\clearpage


\chapter{Looking Forward}

\section{Features}

\section{Analytics}
\clearpage


\chapter{Conclusions}

\section{Looking forward}

\subsection{Features}

\subsection{Analytics}

%%%

\section{Proposal to the University}

%%%

\section{Acknowledgments}

%%%

\section{Resources}

\subsection*{Source code}

The source code for Skedge is available online under an open source license:\\
\url{https://github.com/RocHack/skedge}.

\subsection*{Live site}

\noindent The site can be found at \url{http://skedgeur.com} or \url{http://skedge.org} (very recently acquired, simply redirects to the former).
\clearpage

\begin{thebibliography}{1}

	\bibitem{markov} Takis Konstantopoulos {\em Introductory lecture notes on
		Markov Chains and Random Walks.} Uppsala University,
		\\\url{http://www2.math.uu.se/~takis/L/McRw/mcrw.pdf}

\end{thebibliography}

\addcontentsline{toc}{chapter}{Bibliography}

\clearpage
\section*{Appendix}

\end{document}
