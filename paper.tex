\documentclass[titlepage]{report}

\usepackage{setspace}

\usepackage[utf8]{inputenc}
\usepackage{amsmath}   
\usepackage{mathtools}
\usepackage{graphicx}
\usepackage{subcaption}
\usepackage{float}
\usepackage{hyperref}
\usepackage{listings}
\usepackage[labelfont=bf]{caption}
\usepackage[margin=3.5cm]{geometry}

%\usepackage{qtree}
%\usepackage{tikz}
%\usepackage{tikz-qtree}

%\usepackage[]{algorithm2e}

\lstset{
	basicstyle=\footnotesize\ttfamily,
}

\title{
\vspace{80pt}\\
\LARGE \bfseries SKEDGE
\\\vspace{10pt}\Large Smarter course scheduling for our\\University of Rochester
}

\author{
	Dan Hassin\\
    \vspace{5pt}\\
    Supervised by Professor Philip Guo\\
    \vspace{5pt}\\
    Department of Computer Science\\
    University of Rochester\\
    Rochester, New York\\
}

\date{April 12, 2016\\
    \vspace{150pt}\\
    submitted in partial fulfillment of\\
    the requirements for the degree\\
    honors bachelor of science\\
}

\begin{document}

\maketitle

\doublespacing

% Direct search for specific course
% Discovery, browsing, exploring
% behavioral patterns

% STAT: Find major of users, find how many non-major courses they found

\begin{abstract}

In this paper I present Skedge, a web application for students to comfortably and effectively engage with the University's course catalog. Skedge matches and surpasses the capabilities of the existing University tool for this purpose, ``Course Description / Course Schedule'' (CDCS) and presents its information in a more visually pleasing way. As a result, Skedge boasts strong user-retention rates, long session durations, and high student adoption despite having virtually no advertisement. Through collected usage data, I demonstrate that a) Skedge's differences from and additions to CDCS are usable and have real need, b) the two major use-cases associated with course browsing---direct search and exploratory search---are effectively accommodated by Skedge, and c) Skedge's search mechanism is user-friendly and self-teaches to users over time.

\end{abstract}

\tableofcontents


\chapter{Introduction}

This paper will begin by discussing Markov chains and how we set out to use them for our goal of creating an assistive program editor. We will then describe in detail how we implemented a Markov chain to house program syntax trees, how we were able to generate code using it, and finally we'll discuss our results and address the limitations of our product.

\section{Introduction}

\subsection{Markov chain}

A Markov chain is a mathematical system that undergoes a stochastic process whereby


\begin{figure}[ht]
    \centering
        \begin{subfigure}[h]{14cm}
            \centering
            \includegraphics[width=1.00\textwidth]{images/cdcs}
        \end{subfigure}\\
        \vspace{20pt}\\
        \begin{subfigure}[h]{15cm}
            \centering
            \includegraphics[width=1.00\textwidth]{images/skedge}
        \end{subfigure}
        \vspace{20pt}\\
    \caption{CDCS (top) and Skedge (bottom) for the search query {\tt csc}.}
\end{figure}


\clearpage

\include{02-grievances}
\clearpage


\chapter{Data Analytics}

Hypotheses:

1. Skedge's differences from and additions to CDCS are usable and have real need

2. Skedge’s navigations-per-add and other metrics demonstrate effectiveness of the use cases
a) direct searching, and
b) course browsing

3. Skedge’s DSL is user-friendly; users learn more advanced search types over time by using it

\section{Usage}

\subsection{General}

Since November 3rd 2015 (137 days)
3,768 unique users
4,500 schedules
Average 90 sessions/day
Average 4.92 pages/session
Average 5:31 minutes/session
28\% of sessions are from new users

MOBILE RESULT

\subsection{Search}

% figs

  \subsubsection{Empty searches}

  Can learn from these
  Some funny ones

\subsection{Course blocks}

40\% of sessions have at least one block-click
Average of 4.94 block-clicks per session

\subsection{Social}

90 users have linked Skedge to Facebook
Since March 1st,
4,000+ visits (200 visits/day)
~60\% of visits to /social were returning visitors
90 overlays onto friends’ schedules
10 clicks to Facebook profiles :(
- get stats from the fb dashboard

\subsection{Conclusion}

Success! Considering skedge is OPTIONAL.
+ course blocks (obv usecase, can't click)
+ exports (not supported by thing)
+ mobile

\section{Navigations-per-add}

\subsection{Definitions}

A navigation is defined as
a search, or
a click on an instructor’s name, or
a click on a crosslisted or prerequisite course link

The navigations-per-{add, bookmark} measure is
the number of navigations a user took (within one session) until a course was {added, bookmarked}

\subsection{Trends}

%figs

\subsection{Breaking them apart}

  behavioral patterns
  Direct search for specific course
  Discovery, browsing, exploring

  \subsubsection{Direct searches}

  \subsubsection{Browse}

  % STAT: Find major of users, find how many non-major courses they found

\subsection{Conclusion}

%figs

Effective++

\section{Users' search types over time}

\subsection{Definitions}

Points for search by (omits number and dept.):

description
credits
crosslisted
CRN
instructor
title
year
term
‘random’
upper-level writing

“CSC” → 0
“MTH 165” → 0
“taught by hema” → 1   ✓    (2 searches) 
“random mur 1-2 credits” → 2   ✓    (1 search) 


\subsection{Trends}

% fig

First increase (60.5% of users)
Median: 2 searches
Average: 4.23 searches
(Starting at 1 counts as an increase value of 0)

Second increase (7.9% of users)
Median: 8 searches
Average: 17.52 searches

\subsection{Conclusion}

DSL++
\clearpage

\include{04-conclusion}
\clearpage

\begin{thebibliography}{1}

	\bibitem{esprima} Ariya Hidayat {\em Esprima}
		\\\url{http://esprima.org/}

	\bibitem{markov} Takis Konstantopoulos {\em Introductory lecture notes on
		Markov Chains and Random Walks.} Uppsala University,
		\\\url{http://www2.math.uu.se/~takis/L/McRw/mcrw.pdf}

	\bibitem{hosa} Frank Pfenning, Conal Elliott
		{\em Higher-Order Abstract Syntax.}
		ACM SIGPLAN Notices. Vol. 23. No. 7. ACM, 1988.
		\\\url{http://www.cs.cmu.edu/~fp/papers/pldi88.pdf}

	\bibitem{escodegen} Yusuke Suzuki {\em Escodegen}
		\\\url{https://github.com/Constellation/escodegen}

	\bibitem{parser api} Mozilla Developer Network {\em SpiderMonkey Parser API}
		\\\url{https://developer.mozilla.org/en-US/docs/SpiderMonkey/Parser_API}

\end{thebibliography}

\addcontentsline{toc}{part}{Bibliography}

\clearpage
\section*{Appendix}

\end{document}
